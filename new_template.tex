\documentclass[a4paper,11pt]{article}

% Pacchetti essenziali
\usepackage[utf8]{inputenc}
\usepackage[T1]{fontenc}
\usepackage[italian]{babel}
\usepackage{lmodern}

% Matematica
\usepackage{amsmath, amssymb, amsthm}
\usepackage{mathtools}  % per equazioni più avanzate

% Grafica e colori
\usepackage{graphicx}
\usepackage{xcolor}

% Box evidenziati (molto utili per riassunti)
\usepackage{tcolorbox}
\tcbuselibrary{skins, theorems}

% Margini
\usepackage[a4paper, margin=2cm]{geometry}

% Intestazioni e piè di pagina
\usepackage{fancyhdr}
\pagestyle{fancy}
\fancyhf{}
\rhead{\thepage}
\lhead{\nouppercase{\leftmark}}

% Definizione di ambienti personalizzati
\newtcbtheorem[number within=section]{definizione}{Definizione}{
  colback=green!5,
  colframe=green!35!black,
  fonttitle=\bfseries,
  separator sign none}{def}

\newtcbtheorem[no counter]{teorema}{Teorema}{
  colback=blue!5,
  colframe=blue!35!black,
  fonttitle=\bfseries}{thm}

\newtcbtheorem[no counter]{esempio}{Esempio}{
  colback=yellow!10,
  colframe=orange!70!black,
  fonttitle=\bfseries}{ex}

\newtcbtheorem[no counter]{nota}{Nota}{
  colback=gray!10,
  colframe=gray!50!black,
  fonttitle=\bfseries}{nota}

% Titolo e metadati
\title{\textbf{Riassunto di Sistemi Dinamici}}  % Cambia il titolo
\author{Tuo Nome}
\date{\today}

\begin{document}

\maketitle
\tableofcontents
\newpage

\section{Introduzione ai Sistemi Dinamici}

\begin{definizione}{Sistema dinamico}{}
Un sistema dinamico è un modello matematico che descrive l'evoluzione temporale di un sistema fisico o astratto attraverso equazioni differenziali o mappe.
\end{definizione}

\begin{teorema}{Teorema di esistenza e unicità}{}
Sia $f(t, x)$ continua in $t$ e localmente lipschitziana in $x$. Allora, per ogni $(t_0, x_0)$, esiste un intervallo temporale in cui la soluzione di
\[
\dot{x} = f(t, x), \quad x(t_0) = x_0
\]
è unica.
\end{teorema}

\section{Equazioni differenziali lineari}

Le equazioni differenziali lineari a coefficienti costanti hanno la forma:
\[
\dot{\mathbf{x}} = A \mathbf{x} + \mathbf{b}(t)
\]
dove $A$ è la matrice di sistema.

La soluzione generale è:
\[
\mathbf{x}(t) = e^{At} \mathbf{x}(0) + \int_0^t e^{A(t-\tau)} \mathbf{b}(\tau) \, d\tau
\]

\begin{esempio}{Sistema massa-molla-smorzatore}{}
Consideriamo il sistema:
\[
m \ddot{x} + c \dot{x} + k x = 0
\]
Equazione caratteristica: $m s^2 + c s + k = 0$.
\end{esempio}

\section{Diagrammi di blocco e funzioni di trasferimento}

La funzione di trasferimento in dominio di Laplace è:
\[
G(s) = \frac{Y(s)}{U(s)} = \frac{b_m s^m + \cdots + b_0}{s^n + a_{n-1} s^{n-1} + \cdots + a_0}
\]

\begin{nota}{}{}
Per sistemi in retroazione negativa: $T(s) = \frac{G(s)}{1 + G(s)H(s)}$.
\end{nota}

\section{Analisi di stabilità}

\begin{teorema}{Criterio di Routh-Hurwitz}{}
Un sistema è asintoticamente stabile se tutti i termini della prima colonna della tabella di Routh sono positivi.
\end{teorema}

\section{Spazio degli stati}

Rappresentazione:
\[
\begin{cases}
\dot{\mathbf{x}} = A \mathbf{x} + B \mathbf{u} \\
\mathbf{y} = C \mathbf{x} + D \mathbf{u}
\end{cases}
\]

Matrice di trasferibilità: $G(s) = C(sI - A)^{-1}B + D$.

\end{document}
