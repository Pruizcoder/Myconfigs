\documentclass[a4paper,15pt]{article}

% Packages
\usepackage{amsmath, amssymb}   % Math symbols
\usepackage{graphicx}           % For images
\usepackage{tikz}               % For drawings
\usepackage{geometry}           % Page layout
\usepackage{fancyhdr}           % Headers and footers
\usepackage{datetime2}          % Date and time

% Page Layout
\geometry{margin=1in}
\pagestyle{fancy}
\fancyhf{}
\lhead{\DTMnow}  % Current date and time
\cfoot{\thepage}

% Custom Commands
\newcounter{lecture}
\setcounter{lecture}{1}
\newcommand{\lecture}[1]{
    \section*{Lecture \thelecture: #1}
    \addtocounter{lecture}{1}
    \noindent\textbf{Date:} \today \quad \textbf{Time:} \DTMcurrenttime
    \vspace{0.5cm}
    \hrule
    \vspace{0.5cm}
}

% Important Note Environment
\newcommand{\important}[1]{
    \vspace{0.3cm}
    \noindent\textbf{\underline{Important:}} #1
    \vspace{0.3cm}
}

\begin{document}

\lecture{Introduzione al corso di reti logice e architettura dei calcolatori}
Durante il corso verranno trattati diversi "livellic"
\itemize
\item FISICO: reticoli cristallini, elettroni, dove si creano i componenti fisici(non trattati nel corso
\item ELETTRICO: circuiti, cavi (non trattati nel corso)
\item LOGICO: valori logici \textbf{astratti}
\item ARCHITETTURALE: sviluppo a grandi "pezzi" dell' architettura
\\\\
* più scendiamo di astrazione più aumenta l'accuratezza delle valutazioni(e l'impegno)
\\
* più saliamo sarà l'efficacia di eventuali modifiche
\subsection{Algebra Booleana}
L'algebra booleana ha 2 valori: 0 e 1 vero e falso(true and false)
\begin{table}
    \begin{tabular}{lll}
        0 & 0 & 0 \endarray
    
\end{tabular}
\end{table}
\end{document}
